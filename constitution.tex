\documentclass{scrartcl}

\usepackage{blindtext}
\usepackage{hyperref}

\hypersetup{
    colorlinks=true,
    linkcolor=blue,
    filecolor=magenta,
    urlcolor=cyan,
    pdftitle={Lancaster University Computer Science Society Constitution},
    pdfpagemode={UseOutlines},
    breaklinks=true,
}

\urlstyle{same}

\title{Lancaster University Computer Science Society Constitution}
\author{Jonathan Leeming}
\date{March 2023}


\begin{document}
    \maketitle

    \clearpage
    \tableofcontents

    \clearpage
    \section{Preface}
        \label{preface}
        \subsection{Format}
            \label{preface--format}
            All sections of this document are to be considered normative unless otherwise specified.

        \subsection{Definitions}
            \label{preface--definition}
            This section is non-normative.

            \begin{enumerate}
                \item The "AGM": the Annual General Meeting.
                \item An "EGM": an Extraordinary General Meeting.
                \item The "Executive": the Executive Committee of the society.
                \item "Impeaching Members": members of the Society invoking the Articles of Impeachment (\ref{executive--impeachment}).
                \item The "Membership": the collection of persons holding membership of the Society in any kind.
                \item "SCC": \href{https://www.lancaster.ac.uk/scc}{Lancaster University School of Computing and Communications}.
                \item The "SU", or the "Union": \href{https://www.lancastersu.co.uk}{Lancaster University Student's Union}.
                \item The "University": \href{https://www.lancaster.ac.uk}{Lancaster University}.
            \end{enumerate}

    \clearpage
    \section{Documentation}
        \label{documentation}
        \subsection{Applicable Law}
            \label{documentation--applicable-law}
            \begin{enumerate}
                \item This and all other documents produced by the society are superseded by applicable law.
                \item In the event of a contradiction between applicable law and any other documentation the Society may use, the applicable law takes precedence.
                \item In the event of a contradiction between multiple applicable laws, advice should be sought from one or more of the Union, SCC, or a professional legal advisor (such as a solicitor).
            \end{enumerate}


        \subsection{Union Documentation}
            \label{documentation--union}
            \begin{enumerate}
                \item This and all other documents produced by the society are superseded by Union documents on Governance and Bye-laws, in addition to any and all other documents pertaining to the governance and regulation of societies.
                \item In the event of a contradiction between Union documentation and documentation produced by the society, Union documentation is to take precedence.
                \item In the event of a contradiction within Union documentation, advice should be sought from the Union.
            \end{enumerate}

        \subsection{SCC Documentation}
            \label{documentation--scc}
            \begin{enumerate}
                \item This and all other documents produced by the society are superseded by SCC policy, organisation, and rules.
                \item In the event of a contradiction between SCC documentation and documentation produced by the society, SCC documentation is to take precedence.
                \item In the event of a contradiction within SCC documentation, advice should be sought from SCC.
                \item In the event of a contradiction between Union and SCC documentation, advice should be sought from both the Union and SCC.
            \end{enumerate}

        \subsection{The Constitution}
            \label{documentation--constitution}
            \begin{enumerate}
                \item This, the constitution, is the ultimate authority on matters pertaining to the policy, organisation, and rules of the society, aside from the exemptions provided in \ref{documentation--applicable-law}, \ref{documentation--union}, and \ref{documentation--scc}.
                \item In the event of a contradiction within the constitution, resolution shall be voted upon by the Upper Cabinet (\ref{executive--cabinet--upper}). This resolution shall be published as a Presidential Decree (\ref{documentation--presidential-decree}), and a constitutional amendment should be proposed at the next opportunity.
                \item This, the constitution, may be altered in substance only by passing an amendment at a General Meeting, as detailed in \ref{gm--consitutional-amendment}.
                \item This, the constitution, may be altered in style or aesthetic only by the Secretary with the authorisation of the entire Upper Cabinet and a simple majority of the Lower Cabinet (\ref{executive--cabinet--lower}).
            \end{enumerate}

        \subsection{Bye-Laws}
            \label{documentation--bye-law}
            \begin{enumerate}
                \item Bye-laws are superseded both by the constitution and any documents superior to the constitution.
                \item In the event of a contradiction between or within bye-laws, resolution shall be voted upon by the Executive. This resolution should be published as a Presidential Decree (\ref{documentation--presidential-decree}), and an amendment to the applicable bye-laws should be proposed at the next opportunity.
                \item Bye-laws may be created, revised, or repealed only by passing the amendment at a Bye-Election, as detailed in \ref{bye-election--bye-law-amendment}.
            \end{enumerate}

        \subsection{Directives}
            \label{documentation--directive}
            \begin{enumerate}
                \item Directives are superseded both by bye-laws and any documents superior to bye-laws, including the constitution.
                \item In the event of a contradiction between or within directives, resolution shall be determined in consultation with the original authors. In the event this is not possible, resolution shall be determined in consultation with the executive members holding the office of the original authors. All applicable directives should be amended with this resolution at the next available opportunity.
                \item Directives may be created, revised, or repealed by passing a simple majority vote of the Cabinet.
                \item Directives may be proposed only by members of the Executive Committee.
            \end{enumerate}

        \subsection{Presidential Decrees}
            \label{documentation--presidential-decree}
            \begin{enumerate}
                \item Presidential Decrees are documents of equivalent standing to the constitution itself.
                \item In the event of a contradiction between the Constitution and a Presidential Decree, the Presidential Decree takes precedence, with the exception of the Articles of Impeachment (\ref{executive--impeachment}), which take precedence.
                \item All Presidential Decrees expire upon the departure of the signatory of said Decrees from office.
                \item A Presidential Decree may be extended with bestowal of the seal of the incoming President.
                \item Presidential Decrees may be created, revised, or repealed only by the incumbent President.
            \end{enumerate}

    \clearpage
    \section{General Status}
        \label{general}
        \subsection{Identification}
            \label{general--identification}
            \begin{enumerate}
                \item This society shall be called the "Lancaster University Computer Science Society".
                \item This society may be referred to either as "LUCompSoc" or simply "CompSoc".
                \item This society is categorised as an Academic Society.
                \item The proceedings of this society should fall within the Aims of the Society, as described in Article 2 of Bye-Law 2023.03.000.
            \end{enumerate}

        \subsection{Governance}
            \label{general--governance}
            \begin{enumerate}
                \item The student body of this society is governed jointly by the Union, SCC, and the Executive (\ref{executive}).
            \end{enumerate}

    \clearpage
    \section{Membership}
        \label{membership}
        \subsection{Qualification}
            \label{membership--qualification}
            \begin{enumerate}
                \item Membership of the society is available to all persons, either presently or formerly associated with either the Union or the University.
                \item Membership of the society may either be Full, Associate, or Honorary.
                \item Honorary membership may be granted either for life or for a set period of time, subject to the agreement of a qualified majority at a General Meeting. This privilege shall be granted normally to former members for outstanding service to the society.
            \end{enumerate}

        \subsection{Rights}
            \label{membership--rights}
            \begin{enumerate}
                \item All members may attend and speak at general meetings.
                \item All members may participate in all society activities.
                \item Any Full member may propose a motion at a general meeting.
                \item Any Full member may vote at a general meeting.
                \item All Full members have the right to stand for all positions in elections.
                \item All Full members have the right to nominate or second a candidate in elections.
                \item Honorary members share all the rights of Associate Members.
            \end{enumerate}

        \subsection{Exclusion}
            \label{membership--exclusion}
            \begin{enumerate}
                \item The Executive reserves the right to refuse or revoke membership, with the approval of the Activities Council, of an individual for breaching this, the Constitution, the Bye-Laws of the society, the Safety Framework of the Union, or for bringing the society into disrepute, subject to the Complaints and Appeals procedure 
            \end{enumerate}

        \subsection{Fees}
            \label{membership--fees}
            \begin{enumerate}
                \item The Executive Committee may, at its discretion, fix a membership fee for Full and Associate members.
                \item Honorary membership shall not have an associated fee.
                \item The Executive Committee, and the Assembly of a General Meeting (\ref{gm--assembly}), shall each have the right to offer free or discounted membership for a period of no more than one year to any person as they see fit. There is no limit to the number of times to which an individual may be bestowed this honour.
            \end{enumerate}

    \clearpage
    \section{The Executive Committee}
        \label{executive}
        \begin{enumerate}
            \item The Executive is the primary governing body of the Society.
            \item The Executive is deemed to be in session during University term time.
            \item No member may be compelled to perform duties of any kind when the Executive is not in session.
        \end{enumerate}

        \subsection{Positions}
            \label{executive--positions}
            \subsubsection{President}
                \label{executive--positions--president}
                \begin{enumerate}
                    \item The President shall be the primary representative of the society to the Union, to the University, and to external bodies.
                    \item The President shall fulfil all the duties demanded of them by the Union.
                    \item The President shall co-ordinate and oversee the activities both of the Executive and of the society as a whole.
                    \item The President is empowered to delegate any and all duties and responsibilities to other members of the Executive.
                    \item The President shall attend all committees and meetings of which they are a member ex officio.
                    \item The President shall ensure the society is represented at all applicable meetings.
                    \item The President shall be responsible for the running of the Executive.
                    \item The President is empowered to make decisions on behalf of the Executive.
                    \item The Executive may call the President to account for any decisions made on its behalf.
                    \item The President shall be responsible for maintaining the reputation of the society, of the members of the society, of the facilities of the society, including the Societal Room.
                    \item The President shall act as Chair to all society meetings.
                    \item The President shall act as Returning Officer at all elections, acting as a guardian of the democratic process, and assisting in the peaceful transition of power.
                    \item The President shall be a Full member of the society, having held office within the Executive prior to ascension to the office of President.
                    \item The President shall be elected by a Qualified Majority of the Membership.
                    \item The President shall have the right to vote in Executive Meetings.
                    \item Having held the office of President, no member again this rank shall attain.
                \end{enumerate}

            \subsubsection{Vice-President}
                \label{executive--positions--vice-president}
                \begin{enumerate}
                    \item The Vice-President shall assist the President in the completion of Presidential duties.
                    \item The Vice-President shall assist the President in liaising with the Oversight Committee, the Union, and the University.
                    \item The Vice-President shall co-ordinate with the Welfare Officer to ensure both the representation and promotion of issues, both welfare and academic in nature.
                    \item In the absence of the President, the Vice-President shall speak with Presidential authority.
                    \item The Vice-President shall be a Full member of the society.
                    \item The Vice-President shall be elected by a Qualified Majority of the Membership.
                    \item The Vice-President shall have the right to vote in Executive Meetings.
                \end{enumerate}

            \subsubsection{Secretary}
                \label{executive--positions--secretary}
                \begin{enumerate}
                    \item The Secretary shall be jointly responsible with the Treasurer for the administration of all manners relating to memberships and subscriptions.
                    \item The Secretary shall be responsible for all Society correspondence and administration.
                    \item The Secretary shall be responsible for taking minutes at meetings.
                    \item The Secretary shall be responsible for the distribution of agendas and minutes of meetings to the Membership.
                    \item The Secretary shall be responsible for the distribution of non-confidential information pertaining to the organisation of the Society to the Membership.
                    \item The Secretary shall, with the Treasurer, be jointly responsible for the administration of all manners relating to memberships and subscriptions.
                    \item The Secretary shall be responsible for ensuring any and all public documentation is accurate and current.
                    \item The Secretary may, at any time, be given, by unanimity of the Executive, the authority to amend the style and aesthetic of the Constitution.
                    \item The Secretary shall, with the Education and Socials Officers, be jointly responsible for the organisation of social events and activities with Industry Contacts.
                    \item The Secretary shall act an ambassador for the Society. This is to include, but not be limited to, the initiation of relationships with and co-ordination with current and future Industry Contacts.
                    \item The Secretary shall be a Full member of the Society.
                    \item The Secretary shall be elected by a Qualified Majority of the Membership.
                    \item The Secretary shall have the right to vote in Executive Meetings.
                \end{enumerate}

            \subsubsection{Treasurer}
                \label{executive--positions--treasurer}
                \begin{enumerate}
                    \item The Treasurer shall maintain the society finances to be transparent and well-ordered.
                    \item The Treasurer shall ensure that all financial regulations, of the Union, of the University, and of applicable law, are adhered to.
                    \item The Treasurer shall maintain a current account of all transactions in which the society partakes.
                    \item The Treasurer shall advise the Executive and the Assembly as to the finances of the society.
                    \item The Treasurer shall present a Statement of Revenue and Expenditure to the AGM, and there shall it be ratified.
                    \item The Treasurer shall be charged with creation of Fund Acquisition Methods as appropriate to ensure the future solvency of the society.
                    \item The Treasurer shall, with the Secretary, be jointly responsible for the administration of all manners relating to memberships and subscriptions.
                    \item The Treasurer shall act as an ambassador for society in an all financial matters. This is to include, but not be limited to, the initiation of relationships with and co-ordination with current and future sponsors.
                    \item The Treasurer shall be a Full member of the Society.
                    \item The Treasurer shall be elected by a Qualified Majority of the Membership.
                    \item The Treasurer shall have the right to vote in Executive Meetings.
                \end{enumerate}

            \subsubsection{Commissioner for Continuity}
                \label{executive--positions--continuity-commissioner}
                \begin{enumerate}
                    \item The Commissioner for Continuity shall advise the Executive on the operation and organisation of the Society.
                    \item The Commissioner for Continuity shall be elected by a simple majority of the Cabinet.
                    \item The Commissioner for Continuity shall not have the right to vote in Executive Meetings, except in the event of a tie amongst the Executive, in which case the Commissioner for Continuity shall have the vote to break said tie.
                    \item The Commissioner for Continuity shall be a Full member of the Society, having held office within the Upper Cabinet prior to ascension to the Commission.
                \end{enumerate}

            \subsubsection{Education Officer}
                \label{executive--positions--education-officer}
                \begin{enumerate}
                    \item The Education Officer shall be responsible for the organisation of Educational Events, as defined in Section 1 of Article 3 of Bye-Law 2023.03.000.
                    \item The Education Officer shall, with the Welfare Officer, be jointly responsible for liaison with SCC directly on behalf of both the Society and the Membership.
                    \item The Education Officer is empowered to deputise the Technical Officer and, with the authorisation of said Technical Officer, a Technical Director to be responsible for an Educational Event.
                    \item The Education Officer shall solely have the authority to appoint and dismiss as many Education Directors as is seen to be appropriate.
                    \item The Education Officer shall be elected by a Qualified Majority of the Membership.
                    \item The Education Officer shall have the right to vote in Executive Meetings.
                \end{enumerate}

            \subsubsection{Events Officer}
                \label{executive--positions--events-officer}
                \begin{enumerate}
                    \item The Events Officer shall be responsible for the organisation of large-scale events, in co-ordination with other applicable members of the Executive.
                    \item The Events Officer shall be responsible for the creation of success criteria for each event.
                    \item The Events Officer shall be empowered to deputise any member as is seen to be appropriate to ensure the success of the event.
                    \item The Events Officer shall be elected by a Qualified Majority of the Membership.
                    \item The Events Officer shall have the right to vote in Executive Meetings.
                \end{enumerate}

            \subsubsection{Publicity Officer}
                \label{executive--positions--publicity-officer}
                \begin{enumerate}
                    \item The Publicity Officer shall be responsible for routine correspondence, both internal and external.
                    \item The Publicity Officer shall be responsible for the publication and proclamation of all appropriate matters to the Membership.
                    \item The Publicity Officer shall be responsible for the publicisation of all events and campaigns at least one week prior to the occurrence of the same, or, in the eventuality exceptional circumstances prevail, as soon after this deadline as is reasonably possible.
                    \item The Publicity Officer shall be responsible for the currency and relevancy of both the content and platform of all social network groups and news feeds utilised by the Society.
                    \item The Publicity Officer shall be responsible for the production and organisation of all visual and aural (including multi-media) materials for the promotion of Society events and campaigns.
                    \item The Publicity Officer shall be responsible for the maintenance and procurement of publicity supplies.
                    \item The Publicity Officer shall be responsible for the maintenance of documentation pertaining to publicity materials, including, but not being limited to,
                        \begin{enumerate}
                            \item stylistic materials (including, but not being limited to, colour schemes, font palettes, and icon sets),
                            \item templates, and
                            \item physical materials (including, but not being limited to, banners, posters, and leaflets).
                        \end{enumerate}
                    \item The Publicity Officer shall be elected by a Qualified Majority of the Membership.
                    \item The Publicity Officer shall have the right to vote in Executive Meetings.
                \end{enumerate}

            \subsubsection{Socials Officer}
                \label{executive--positions--socials-officer}
                \begin{enumerate}
                    \item The Socials Officer shall be responsible for the organisation of Social Events, as defined in Section 2 of Article 3 of Bye-Law 2023.03.000.
                    \item The Socials Officer shall, in consultation with the Welfare Officer, be responsible for the foster of inclusivity at Social Events to the greatest practicality.
                    \item The Socials Officer shall, with the Publicity Officer, be jointly responsible for the promotion of the non-educational aspects of the Society.
                    \item The Socials Officer shall act as Sober Officer for all events having such requirements, unless written notice is given to the President prior affirming the desire of another member of the Cabinet to take the role.
                    \item The Socials Officer shall be elected by a Qualified Majority of the Membership.
                    \item The Socials Officer shall have the right to vote in Executive Meetings.
                \end{enumerate}

            \subsubsection{Technical Officer}
                \label{executive--positions--technical-officer}
                \begin{enumerate}
                    \item The Technical Officer shall be responsible for the co-ordination, maintenance, and reliability of all technical systems and equipment utilised by the Society.
                    \item The Technical Officer shall, in consultation with the Treasurer, be responsible for the procurement of equipment utilised by the Society.
                    \item The Technical Officer shall be responsible for the oversight of and provision of resources to projects undertaken by the Society.
                    \item The Technical Officer shall be responsible for the completion of a Feasibility Analysis of project proposals.
                    \item The Technical Officer shall advise the Executive as to the feasibility of project proposals.
                    \item The Technical Officer shall facilitate the internship of members to projects.
                    \item The Technical Officer shall solely have the authority to appoint and dismiss as many Technical Directors as is seen to be appropriate.
                    \item The Technical Officer shall be elected by a Qualified Majority of the Membership.
                        \subitem In this process, the successful Technical Officer should convince said membership as to the presence of such competencies as are necessary for the role.
                    \item The Technical Officer shall have the right to vote in Executive Meetings.
                \end{enumerate}

            \subsubsection{Welfare Officer}
                \label{executive--positions--welfare-officer}
                \begin{enumerate}
                    \item The Welfare Officer shall be responsible for the stewardship of the wellbeing and welfare of the Membership.
                    \item The Welfare Officer shall be ensure all Society activities conform both to the Safety Framework of the Union and to the Code of Safe Practice of the Society as detailed in Bye-Law 2023.03.001.
                    \item The Welfare Officer shall be responsible for liaison with SCC for the purpose of the promotion of the wellbeing and welfare of the Membership, including, but not being limited to, the proclamation of the complaints of the Membership pertaining to the courses offered by SCC.
                    \item The Welfare Officer shall solely have the authority to appoint and dismiss as many Welfare Directors as is seen to be appropriate.
                    \item The Welfare Officer shall be elected by a Qualified Majority of the Membership.
                    \item The Welfare Officer shall have the right to vote in Executive Meetings.
                \end{enumerate}

            \subsubsection{Education Director}
                \label{executive--positions--education-director}
                \begin{enumerate}
                    \item Education Directors shall assist the Education Officer in the execution of the duties of the office.
                    \item Education Directors shall be appointed and dismissed only by the Education Officer.
                    \item Education Directors shall not have the right to vote in Executive Meetings.
                \end{enumerate}

            \subsubsection{Technical Director}
                \label{executive--positions--technical-director}
                \begin{enumerate}
                    \item Technical Directors shall assist the Technical Officer in the execution of the duties of the office.
                    \item Technical Directors shall be appointed and dismissed only by the Technical Officer.
                    \item Technical Directors shall not have the right to vote in Executive Meetings.
                \end{enumerate}

            \subsubsection{Welfare Director}
                \label{executive--positions--welfare-director}
                \begin{enumerate}
                    \item Welfare Directors shall assist the Welfare Officer in the execution of the duties of the office.
                    \item Welfare Directors shall be appointed and dismissed only by the Welfare Officer.
                    \item Welfare Directors shall not have the right to vote in Executive Meetings.
                \end{enumerate}

            \subsubsection{Executive Advisor}
                \label{executive--positions--advisor}
                \begin{enumerate}
                    \item Executive Advisors shall advise the Executive as to the operation and organisation of the Society.
                    \item Executive Advisors may be appointed to assist with Society Operations including, but not being limited to, Alumni Relations, publicity, and recruitment.
                    \item Executive Advisors shall not have the rights to vote in Executive Meetings.
                \end{enumerate}

        \subsection{The Cabinet}
            \label{executive--cabinet}
            \begin{enumerate}
                \item Within the Executive shall be the Cabinet, divided into two houses of seniority: the Upper Cabinet and the Lower Cabinet.
                \item All members of the Cabinet shall attend and complete all training offered and required by the Activities Office of the Union.
                \item The Cabinet shall, by simple majority vote, have the authority to appoint and dismiss as many Executive Advisors as are deemed to be appropriate.
            \end{enumerate}

            \subsubsection{The Upper Cabinet}
                \label{executive--cabinet--upper}
                \begin{enumerate}
                    \item The Upper Cabinet shall consist of
                        \begin{enumerate}
                            \item the President,
                            \item the Vice-President,
                            \item the Secretary,
                            \item the Treasurer, and
                            \item the Commissioner for Continuity.
                        \end{enumerate}
                \end{enumerate}

            \subsubsection{The Lower Cabinet}
                \label{executive--cabinet--lower}
                \begin{enumerate}
                    \item The Lower Cabinet shall consist of
                        \begin{enumerate}
                            \item the Education Officer,
                            \item the Events Officer,
                            \item the Publicity Officer,
                            \item the Socials Officer,
                            \item the Technology Officer, and
                            \item the Welfare Officer.
                        \end{enumerate}
                \end{enumerate}

        \subsection{Impeachment}
            \label{executive--impeachment}
            \begin{enumerate}
                \item Any member of the Executive may be impeached for
                    \begin{enumerate}
                        \item negligence of duty;
                        \item gross incompetence;
                        \item serious breach of any safety documentation; or
                        \item posing a grave danger to the Society.
                    \end{enumerate}
                \item The member against whom these Articles of Impeachment are invoked is hereafter referred to as the Defendant.
                \item These Articles of Impeachment may be invoked in any one of the following manners:
                    \begin{enumerate}
                        \item by a single member of the Upper Cabinet against any member of the Executive;
                        \item by a single member of the Lower Cabinet against any member of the Executive, not belonging to the Upper Cabinet;
                        \item by at least two (2) members of the Lower Cabinet against any member of the Upper Cabinet;
                        \item by a single member of the Executive, not belonging to the Cabinet, against a single member of the Executive, not belonging to the Cabinet;
                        \item by at least two (2) members of the Executive, not belonging to the Cabinet, against a single member of the Lower Cabinet;
                        \item by at least three (3) members of the Executive, not belonging to the Cabinet, against a single member of the Upper Cabinet;
                        \item by at least two (2) members of the Membership against a single member of the Executive, not belonging to the Cabinet;
                        \item by at least five (5) members of the Membership against a single member of the Lower Cabinet; or
                        \item by at least ten (10) members of the Membership against a single member of the Lower Cabinet.
                    \end{enumerate}
                \item Notification the invocation of these Articles of Impeachment should be given, in signed and dated writing, to the Executive in the following order, with the most preference being given to the first option, the least to the last:
                    \begin{enumerate}
                        \item to the President, unless either the President is to be impeached or the President is invoking the articles;
                        \item to the Vice-President, unless either the Vice-President is to be impeached or the Vice-President is invoking the articles against the President;
                        \item to the Secretary.
                    \end{enumerate}
                \item Within seven (7) days of receiving notification of the invocation of these Articles of Impeachment, the following must occur:
                    \begin{enumerate}
                        \item the party being impeached must be notified as to the invocation;
                        \item the Executive must be notified as to the invocation;
                        \item the Membership must be notified as to the invocation; and
                        \item an EGM must be called.
                    \end{enumerate}
                \item At the called EGM, the following shall occur:
                    \begin{enumerate}
                        \item the Impeaching Members shall provide argument as to the reason both for impeachment and for removal from office;
                        \item the Defendant shall provide their defence;
                        \item a Referendum shall be taken of the gathered Assembly as to the guilt of the Defendant;
                        \item in the event of a simple majority in favour of the Impeaching Members, the Defendant shall be removed from office;
                        \item a second Referendum shall be taken of the gathered Assembly as to the continued status of membership of the Society of the Defendant;
                        \item in the event of a super-majority of two-thirds of the gathered Assembly in favour, the membership of the Society of the Defendant shall be revoked.
                    \end{enumerate}
                \item The results of impeachment are subject to appeal only to the extent required by Union documentation.
                \item In the event of removal from office, said member shall be disqualified from holding office within the Executive again.
                \item No member may invoke these Articles of Impeachment more than once in a term.
                \item These Articles of Impeachment may be invoked only whilst the Executive is in session.
            \end{enumerate}

    \clearpage
    \section{General Meetings}
        \label{gm}
        \subsection{Assembly}
            \label{gm--assembly}
            \begin{enumerate}
                \item The Assembly at a General Meeting shall represent the totality of the Membership.
                    \subitem The voice of the Assembly shall thus constitute ultimate governing institution of the Society.
                \item All members of the Society shall have the right to be present and vote at all General Meetings.
                \item Decisions made by the Assembly shall be binding to the Executive.
            \end{enumerate}

        \subsection{Constitutional Amendments}
            \label{gm--consitutional-amendment}
            \begin{enumerate}
                \item Amendments to the substance of the Constitution of the Society shall be proposed at a General Meeting.
                \item Proposed amendments shall be ratified and enacted only with the approval of a super-majority of two-thirds of the Assembly.
                \item Prior to referendum, proposed amendments shall be proclaimed to the convened Assembly by the President.
            \end{enumerate}

        \subsection{Dissolution of the Society}
            \label{gm--disolution}
            \begin{enumerate}
                \item The Society may be dissolved at a General Meeting, provided that at least twenty-one (21) days notice of the intention of dissolution has been given to the Membership.
                \item A qualified super-majority of at least two-thirds of the Assembly is required for the Motion of Dissolution to be effective.
                \item In the event of the Dissolution of the Society, the following must occur:
                    \begin{enumerate}
                        \item any debts owed by the Society shall be paid in full;
                        \item any financial assets shall be distributed as required by Union and University policy; and
                        \item the assets of the Society shall be distributed to applicable societies.
                    \end{enumerate}
            \end{enumerate}

        \subsection{The Annual General Meeting}
            \label{gm--agm}
            \begin{enumerate}
                \item Exactly one AGM shall be held each Academic Year in the Lent Term.
                \item An AGM may be called by be called by the following persons:
                    \begin{enumerate}
                        \item by the President;
                        \item by a simple majority of the Upper Cabinet;
                        \item by a simple majority of the Lower Cabinet; or
                        \item by written request of at least five (5) members of the Society.
                    \end{enumerate}
                \item Any request to hold an AGM shall be given, in signed and dated writing, to the Secretary. This request shall then be proposed to the Executive at the next convention of the same.
                \item Reasonable measures shall be undertaken by the incumbent Executive to ensure the timely execution of the request and the viability of participation of all interested members.
                \item At the AGM, the following shall occur:
                    \begin{enumerate}
                        \item any amendments to the Constitution or to the Bye-Laws of the Society shall be proposed and ratified;
                        \item the President shall present a report of the activities of the Society for the prior year;
                        \item the Treasurer shall present a statement of accounts for ratification by the Assembly;
                        \item the Commissioner for Continuity shall announce the Commissioner for the following year, in the absence of a Commissioner, this role shall be filled by the Vice-President; and
                        \item the Executive for the following year shall be elected, following the Election Procedure (\ref{bye-election--election-procedure}).
                    \end{enumerate}
                \item At each AGM, all electable positions shall be deemed to be available for the following year.
                    \subitem In the event a member of the Executive desires to remain in office, said member shall declare candidacy, even in the absence of opposition.
            \end{enumerate}

        \subsection{Extraordinary General Meetings}
            \label{gm--egm}
            \begin{enumerate}
                \item An EGM may be called for a number of reasons including, but not being limited to, invocation of the Articles of Impeachment or an exodus of the incumbent Executive.
                \item Once called, an EGM must be held within fourteen (14) days.
            \end{enumerate}

        \subsection{The General Meeting for the Peaceful Transfer of Power}
            \label{gm--transfer}
            \begin{enumerate}
                \item The General Meeting for the Peaceful Transfer of Power shall be held at the conclusion of the Summer Term.
                \item It is at this meeting that the transfer of power shall occur between the Outgoing and Incoming Executives.
                \item It is at this meeting that an Annual Review shall be presented by the Outgoing Executive to the convened Assembly.
            \end{enumerate}

    \clearpage
    \section{Bye-Elections}
        \label{bye-election}
        \begin{enumerate}
            \item Any proceeding which may occur at a Bye-Election may also occur at either the AGM or at an EGM.
        \end{enumerate}

        \subsection{Election Procedure}
            \label{bye-election--election-procedure}
            \begin{enumerate}
                \item The execution of all Society Elections shall be the responsibility of the President, who shall act as Returning Officer.
                    \subitem In the event that the President declares candidacy for the same elections, the Commissioner for Continuity shall instead take this role.
                    \subitem In the event that the Commissioner for Continuity also declares candidacy for the same elections, a suitably competent and independent member of the Cabinet shall be nominated for the role.
                \item The Membership shall be informed as to the date of the election, available positions, and instruction as to declaration of candidacy at least two weeks prior to the election.
                \item "Re-Open Nominations (RON)" shall be a candidate for all positions in all elections.
                \item Preceding elections shall be speeches from all candidates.
                    \subitem The maximum length of speeches shall be determined in advance by the Returning Officer.
                \item Antecedent to speeches shall be questions put to the candidates by the Assembly.
                \item In the event a candidate is unable to attend due to means beyond the control of the same, written confirmation of intention to stand shall be given to the Returning Officer.
                    \subitem The confirmation of intention to stand may include a speech to be proclaimed to the Assembly by the Returning Officer.
                    \subitem In the event that no such confirmation is received, absent candidates shall be excluded from election.
                \item Each member of the Society shall be entitled to exactly one (1) vote per position.
                    \subitem All members of the Society shall have the right to cast said vote either in person or by proxy.
                    \subitem Votes cast by proxy may be cast either electronically or by delegation to another member of the Society in signed and dated writing.
                \item For each candidate shall a ballot be taken, and the candidate with the greatest number of votes shall duly be elected to the office.
                    \subitem The Returning Officer shall be responsible for the counting of the ballot.
                \item Any and all complaints pertaining to elections shall be brought before the Returning Officer.
                    \subitem In the event of unsatisfactory resolution, complaints shall be escalated to the Chair of the Activities Council of the Union, in compliance with Bye-Law 2023.03.002.
            \end{enumerate}

        \subsection{Bye-Law Amendments}
            \label{bye-election--bye-law-amendment}
            \begin{enumerate}
                \item Amendments to the Bye-Laws of the Society shall be proposed at a Bye-Election.
                \item Proposed amendments shall be ratified and enacted only with the approval of a simple majority of the Assembly.
                \item Prior to referendum, proposed amendments shall be proclaimed to the convened Assembly by the President.
            \end{enumerate}

    \clearpage
    \section{Finance}
        \label{finance}
        \begin{enumerate}
            \item All matters pertaining to Society finance shall be carried out in accordance with the Financial Regulations of the Union and all applicable laws.
            \item Prior the conclusion of the Academic Year, the incumbent and incoming Treasurers shall jointly issue a Statement of Estimation of the Budget of the Society.
                \subitem This Statement shall be presented to the incumbent and incoming Executives for joint ratification by simple majority of the combined committees.
                \subitem Having been ratified, the Statement shall be presented during the Annual Review to the Assembly at the General Meeting for the Peaceful Transfer of Power.
                \subitem This Statement shall be displayed so as to be accessible to all members at any point during the following year.
                \subitem Following the conclusion of the relevant Academic Year, the Statement shall be archived in accordance with Directive 2023.03.001.
            \item Signatories to the Executive Account with the Union or any other applicable financial institution shall ordinarily be
                \begin{enumerate}
                    \item the President or Vice-President,
                    \item the Secretary, and
                    \item the Treasurer.
                \end{enumerate}
                \subitem In the event that any of the above positions remain unfilled, a suitably competent member of the Executive may have bestowed upon them by the Upper Cabinet the authority to be a Signatory of the Society.
                \subitem As part of the assessment of competency, the member in question should complete all applicable financial training from the Financial Controller of the Union.
        \end{enumerate}

    \clearpage
    \section{Projects}
        \label{project}
        \begin{enumerate}
            \item At any point, the Executive may, by simple majority vote, initiate a Project.
            \item Each Project shall be assigned a Technical Director to manage and oversee the same.
            \item Any project lacking both a Director and members to partake in said project shall, with immediate effect, be defunct.
                \subitem A defunct Project shall be restated only by simple majority vote of the Executive.
            \item A Technical Director may, in consultation with the Technical Officer, create Directives to enact policies pertaining to the Projects associated with said Director.
                \subitem Such Directives shall be created in the same manner in which the Executive may create Directives, with the Technical Director acting as President and requiring half the Project Membership to be quorate.
                \subitem The Cabinet shall, by simple majority vote, have Veto Power over Project Directives.
        \end{enumerate}

    \clearpage
    \section{Affiliation and Sponsorship}
        \label{affiliation}
        \begin{enumerate}
            \item The society may affiliate to external bodies in compliance with Section of \href{https://lancastersu.co.uk/resources/articles-of-association-2023/download_attachment}{The Constitution of the Union, Article 47, Section 4}.
            \item Sponsors of the society shall be entitled both to represent and to be represented by the Society, by means either financial or otherwise, as stated in a contract signed jointly by the Financial Signatories of the Society and a representative of the sponsor.
            \item As a minimum, sponsors shall receive the right to affiliate publicly with the Society.
        \end{enumerate}
\end{document}

