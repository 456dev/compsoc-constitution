\documentclass{scrartcl}

\usepackage{blindtext}
\usepackage{hyperref}

\hypersetup{
    colorlinks=true,
    linkcolor=blue,
    filecolor=magenta,
    urlcolor=cyan,
    pdftitle={Lancaster University Computer Science Society Constitution},
    pdfpagemode={UseOutlines},
    breaklinks=true,
}

\urlstyle{same}

\title{Lancaster University Computer Science Society Constitution}
\author{Jonathan Leeming}
\date{March 2023}


\begin{document}
    \maketitle

    \clearpage
    \tableofcontents

    \clearpage
    \section{Preface}
        \label{preface}
        \subsection{Format}
            \label{preface--format}
            All sections of this document are to be considered normative, unless otherwise specified.

        \subsection{Definitions}
            \label{preface--definition}
            This section is non-normative.

            \begin{enumerate}
                \item The "AGM": the Annual General Meeting.
                \item The "Executive": the Executive Committee of the society.
                \item "SCC": \href{https://www.lancaster.ac.uk/scc}{Lancaster University School of Computing and Communications}.
                \item The "SU", or the "Union": \href{https://www.lancastersu.co.uk}{Lancaster University Student's Union}.
                \item The "University": \href{https://www.lancaster.ac.uk}{Lancaster University}.
            \end{enumerate}

    \clearpage
    \section{Documentation}
        \label{documentation}
        \subsection{Applicable Law}
            \label{documentation--applicable-law}
            \begin{enumerate}
                \item This and all other documents produced by the society are superseded by applicable law.
                \item In the event of a contradiction between applicable law and any other document which may be used by the society, the applicable law takes precedence.
                \item In the event of a contradiction between multiple applicable laws, advice should be sought from one or more of the Union, SCC, or a professional legal advisor (such as a solicitor).
            \end{enumerate}


        \subsection{Union Documentation}
            \label{documentation--union}
            \begin{enumerate}
                \item This and all other documents produced by the society are superseded by Union documents on Governance and Bye-laws, in addition to any and all other documents pertaining to the governance and regulation of societies.
                \item In the event of a contradiction between Union documentation and documentation produced by the society, Union documentation is to take precedence.
                \item In the event of a contradiction within Union documentation, advice should be sought from the Union.
            \end{enumerate}

        \subsection{SCC Documentation}
            \label{documentation--scc}
            \begin{enumerate}
                \item This and all other documents produced by the society are superseded by SCC policy, organisation, and rules.
                \item In the event of a contradiction between SCC documentation and documentation produced by the society, SCC documentation is to take precedence.
                \item In the event of a contradiction within SCC documentation, advice should be sought from SCC.
                \item In the event of a contradiction between Union and SCC documentation, advice should be sought from both the Union and SCC.
            \end{enumerate}

        \subsection{The Constitution}
            \label{documentation--constitution}
            \begin{enumerate}
                \item This, the constitution, is the ultimate authority on matters pertaining to the policy, organisation, and rules of the society, aside from the exemptions provided in \ref{documentation--applicable-law}, \ref{documentation--union}, and \ref{documentation--scc}.
                \item In the event of a contradiction within the constitution, resolution shall be voted upon by the Upper Cabinet (\ref{executive--upper-cabinet}). This resolution is to be published as a Presidential Decree (\ref{documentation--presidential-decree}), and a constitutional amendment should be proposed at the next opportunity.
                \item This, the constitution, may be altered in substance only by passing an amendment at the AGM, as detailed in \ref{agm--consitutional-amendment}.
                \item This, the constitution, may be altered in style or aesthetic only by the Secretary with the authorization of the entire Upper Cabinet and a simple majority of the Lower Cabinet (\ref{executive--lower-cabinet}).
            \end{enumerate}

        \subsection{Bye-Laws}
            \label{documentation--bye-law}
            \begin{enumerate}
                \item Bye-laws are superseded both by the constitution and any documents superior to the constitution.
                \item In the event of a contradiction between or within bye-laws, resolution shall be voted upon by the Executive. This resolution should be published as a Presidential Decree (\ref{documentation--presidential-decree}), and an amendment to the applicable bye-laws should be proposed at the next opportunity.
                \item Bye-laws may be created, revised, or repealed only by passing the amendment at a Bye-Election, as detailed in \ref{bye-election--bye-law-amendment}.
            \end{enumerate}

        \subsection{Directives}
            \label{documentation--directive}
            \begin{enumerate}
                \item Directives are superseded both by bye-laws and any documents superior to bye-laws, including the constitution.
                \item In the event of a contradiction between or within directives, resolution shall be determined in consultation with the original authors. In the event this is not possible, resolution shall be determined in consultation with the executive members holding the office of the original authors. All applicable directives should be amended with this resolution at the next available opportunity.
                \item Directives may be created, revised, or repealed by passing a simple majority vote of the Executive Committee.
                \item Directives may be proposed only by members of the Executive Committee.
            \end{enumerate}

        \subsection{Presidential Decrees}
            \label{documentation--presidential-decree}
            \begin{enumerate}
                \item Presidential Decrees are documents of equivalent standing to the constitution itself.
                \item In the event of a contradiction between the Constitution and a Presidential Decree, the Presidential Decree takes precedence, with the exception of the Articles of Impeachment (\ref{executive--impeachment}), which take precedence.
                \item All Presidential Decrees expire upon the departure of the signatory of said Decrees from office.
                \item A Presidential Decree may be extended with bestowal of the seal of the incoming President.
                \item Presidential Decrees may be created, revised, or repealed only by the incumbent President.
            \end{enumerate}

    \clearpage
    \section{General Status}
        \label{general}
        \subsection{Identification}
            \label{general--identification}
            \begin{enumerate}
                \item This society shall be called the "Lancaster University Computer Science Society".
                \item This society may be referred to either as "LUCompSoc", or simply "CompSoc".
                \item This society is categorised as an Academic Society.
                \item This proceedings of this society should fall within the Aims of the Society, as described in Bye-Law 2023.03.000.
            \end{enumerate}

        \subsection{Governance}
            \label{general--governance}
            \begin{enumerate}
                \item The student body of this society is governed jointly by the Union, SCC, and the Executive (\ref{executive}).
            \end{enumerate}

    \clearpage
    \section{Membership}
        \label{membership}
        \subsection{Qualification}
            \label{membership--qualification}
            \begin{enumerate}
                \item Membership to the society is available to all persons either presently or formerly associated with either the Union or the University.
                \item Membership to the society may either be Full, Associate, or Honorary.
                \item Honorary membership may be granted either for life or for a set period of time, subject to the agreement of a qualified majority at the AGM. This privilege shall be granted normally to former members for outstanding service to the society.
            \end{enumerate}

        \subsection{Rights}
            \label{membership--rights}
            \begin{enumerate}
                \item All members may attend and speak at general meetings.
                \item All members may participate in all society activities.
                \item Any Full member may propose a motion at a general meeting.
                \item Any Full member may vote at a general meeting.
                \item All Full members have the right to stand for all positions in elections.
                \item All Full members have the right to nominate or second a candidate in elections.
                \item Honorary members share all the rights of Associate Members.
            \end{enumerate}

        \subsection{Exclusion}
            \label{membership--exclusion}
            \begin{enumerate}
                \item The Executive reserves the right to refuse or revoke membership, with the approval of the Activities Council, of an individual for breaching this, the Constitution, the Bye-Laws of the society, the Safety Framework of the Union, or for bringing the society into disrepute, subject to the Complaints and Appeals procedure 
            \end{enumerate}

        \subsection{Fees}
            \label{membership--fees}
            \begin{enumerate}
                \item The Executive Committee may, at its discretion, fix a membership fee for Full members.
                \item The Executive Committee, and the Assembly of the AGM (\ref{agm--assembly}), shall each have the right to offer free or discounted membership for a period of no more than one year to any person as they see fit. There is no limit to the number of times to which an individual may be bestowed this honour.
            \end{enumerate}

    \clearpage
    \section{The Executive Committee}
        \label{executive}
        \subsection{Positions}
            \label{executive--positions}
            \subsubsection{President}
                \label{executive--positions--president}
                \begin{enumerate}
                    \item The President shall be the primary representative of the society to the Union, to the University, and to external bodies.
                    \item The President shall fulfil all the duties demanded of them by the Union.
                    \item The President shall co-ordinate and oversee the activities both of the Executive and of the society as a whole.
                    \item The President is empowered to delegate any and all duties and responsibilities to other members of the Executive.
                    \item The President shall attend all committees and meetings of which they are a member ex officio.
                    \item The President shall ensure the society is represented at all applicable meetings.
                    \item The President shall be responsible for the running of the Executive.
                    \item The President is empowered to make decisions on behalf of the Executive.
                    \item The Executive may call the President to account for any decisions made on its behalf.
                    \item The President shall be responsible for maintaining the reputation of the society, of the members of the society, of the facilities of the society, including the Societal Room.
                    \item The President shall act as Chair to all society meetings.
                    \item The President shall be a Full member of the society, having held office within the Executive prior to ascension to the office of President.
                \end{enumerate}

            \subsubsection{Vice-President}
                \label{executive--positions--vice-president}
                \begin{enumerate}
                    \item The Vice-President shall assist the President in the completion of Presidential duties.
                    \item The Vice-President shall assist the President in liaising with the Oversight Committee, the Union, and the University.
                    \item The Vice-President shall co-ordinate with the Wellbeing Officer to ensure both the representation and promotion of issues, both welfare and academic in nature.
                    \item In the absence of the President, the Vice-President shall speak with Presidential authority.
                \end{enumerate}

            \subsubsection{Treasurer}
                \label{executive--positions--treasurer}
                \begin{enumerate}
                    \item The Treasurer shall maintain the society finances to be transparent and well-ordered.
                    \item The Treasurer shall ensure that all financial regulations, of the Union, of the University, and of applicable law, are adhered to.
                    \item The Treasurer shall maintain a current account of all transactions in which the society partakes.
                    \item The Treasurer shall advise the Executive and the Assembly as to the finances of the society.
                    \item The Treasurer shall present a Statement of Revenue and Expenditure to the AGM, and there shall it be ratified.
                    \item The Treasurer shall be charged with creation of Fund Acquisition Methods as appropriate to ensure the future solvency of the society.
                    \item The Treasurer shall act as an ambassador for society in an all financial matters. This is to include, but not be limited to, the initiation of relationships with and co-ordination with current and future sponsors and industry contacts, acting solely in the interest of the society.
                \end{enumerate}

        \subsection{The Upper Cabinet}
            \label{executive--upper-cabinet}

        \subsection{The Lower Cabinet}
            \label{executive--lower-cabinet}

        \subsection{Impeachment}
            \label{executive--impeachment}

    \clearpage
    \section{The Annual General Meeting}
        \label{agm}
        \subsection{Assembly}
            \label{agm--assembly}

        \subsection{Constitutional Amendments}
            \label{agm--consitutional-amendment}

    \clearpage
    \section{Bye-Elections}
        \label{bye-election}
        \subsection{Bye-Law Amendments}
            \label{bye-election--bye-law-amendment}

    \clearpage
    \section{Affiliation and Sponsorship}
        \label{affiliation}
        \begin{enumerate}
            \item The society may affiliate to external bodies in compliance with Section of \href{https://lancastersu.co.uk/resources/articles-of-association-2023/download_attachment}{The Constitution of the Union, Article 47, Section 4}
        \end{enumerate}
\end{document}

