\documentclass{article}

\usepackage{blindtext}
\usepackage{hyperref}

\hypersetup{
    colorlinks=true,
    linkcolor=blue,
    filecolor=magenta,
    urlcolor=cyan,
    pdftitle={Lancaster University Computer Science Society Constitution},
    pdfpagemode={UseOutlines},
    breaklinks=true,
}

\urlstyle{same}

\title{Lancaster University Computer Science Society Constitution}
\author{Jonathan Leeming}
\date{March 2023}


\begin{document}
    \tableofcontents

    \clearpage
    \section{Preface}
        \label{preface}
        \subsection{Format}
            \label{preface--format}
            All sections of this document are to be considered normative, unless otherwise specified.

        \subsection{Definitions}
            \label{preface--definition}
            This section is non-normative.

            \begin{enumerate}
                \item \label{preface--definition--executive} The "Executive": the Executive Committee of the society.
                \item \label{preface--definition--scc} "SCC": \href{https://www.lancaster.ac.uk/scc}{Lancaster University School of Computing and Communications}.
                \item \label{preface--definition--su} The "SU", or the "Union": \href{https://www.lancastersu.co.uk}{Lancaster University Student's Union}.
            \end{enumerate}

    \clearpage
    \section{Documentation}
        \label{documentation}
        \subsection{Applicable Law}
            \label{documentation--applicable-law}
            \begin{enumerate}
                \item This and all other documents produced by the society are superseded by applicable law.
                \item In the event of a contradiction between applicable law and any other document which may be used by the society, the applicable law takes precedence.
                \item In the event of a contradiction between multiple applicable laws, advice should be sought from one or more of the Union, SCC, or a professional legal advisor (such as a solicitor).
            \end{enumerate}


        \subsection{Union Documentation}
            \label{documentation--union}
            \begin{enumerate}
                \item This and all other documents produced by the society are superseded by Union documents on Governance and Bye-laws, in addtion to any and all other documents pertaining to the governance and regulation of societies.
                \item In the event of a contradiction between Union documentation and documentation produced by the society, Union documentation is to take precedence.
                \item In the event of a contradiction within Union documentation, advice should be sought from the Union.
            \end{enumerate}

        \subsection{SCC Documentation}
            \label{documentation--scc}
            \begin{enumerate}
                \item This and all other documents produced by the society are superseded by SCC policy, organisation, and rules.
                \item In the event of a contradiction between SCC documentation and documentation produced by the society, SCC documentation is to take precedence.
                \item In the event of a contradiction within SCC documentation, advice should be sought from SCC.
            \end{enumerate}

        \subsection{The Constitution}
            \label{documentation--constitution}
            \begin{enumerate}
                \item This, the constitution, is the ultimate authority on matters pertaining to the policy, organisation, and rules of the society, aside from the exemptions provided in \ref{documentation--applicable-law}, \ref{documentation--union}, and \ref{documentation--scc}.
                \item In the event of a contradiction within the constitution, resolution shall be voted upon by the Upper Cabinet (\ref{executive--upper-cabinet}). This resolution is to be published as a Presidential Decree (\ref{documentation--presidential-decree}), and a constitutional amendment should be proposed at the next opportunity.
                \item This, the constitution, may be altered only by passing an amendment at the AGM, as detailed in \ref{agm--consitutional-amendment}.
            \end{enumerate}

        \subsection{Bye-Laws}
            \label{documentation--bye-law}
            \begin{enumerate}
                \item Bye-laws are superseded both by the constitution and any documents superior to the constitution.
                \item In the event of acontradiction between or within bye-laws, resolution shall be voted upon by the Executive. This resolution should be published as a Presidential Decree (\ref{documentation--presidential-decree}), and an amendment to the applicable bye-laws should be proposed at the next opportunity.
                \item Bye-laws may be created, revised, or repealed only by passing the amendment at a Bye-Election, as detailed in \ref{bye-election--bye-law-amendment}.
            \end{enumerate}

        \subsection{Directives}
            \label{documentation--directive}

        \subsection{Presidential Decrees}
            \label{documentation--presidential-decree}

    \clearpage
    \section{General Status}
        \label{general}
        \subsection{Identification}
            \label{general--identification}
            \begin{enumerate}
                \item This society shall be called the "Lancaster University Computer Science Society".
                \item This society may be refferred to either as "LUCompSoc", or simply "CompSoc".
                \item This society is categorised as an Academic Society.
            \end{enumerate}

        \subsection{Governance}
            \label{general--governance}
            \begin{enumerate}
                \item The student body of this society is governed jointly by the Union, SCC, and the Executive (\ref{executive}).
            \end{enumerate}

    \clearpage
    \section{The Executive Committee}
        \label{executive}
        \subsection{The Upper Cabinet}
            \label{executive--upper-cabinet}

    \clearpage
    \section{The Annual General Meeting}
        \label{agm}
        \subsection{Constitutional Amendments}
            \label{agm--consitutional-amendment}

    \clearpage
    \section{Bye-Elections}
        \label{bye-election}
        \subsection{Bye-Law Amendments}
            \label{bye-election--bye-law-amendment}
\end{document}

